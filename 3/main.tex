\documentclass[10pt, a4paper]{article}

\usepackage{fancyhdr}
\usepackage{extramarks}
\usepackage{amsmath}
\usepackage{amsthm}
\usepackage{amsfonts}
\usepackage{tikz}
\usepackage[plain]{algorithm}
%\usepackage[euler]{textgreek}
\usepackage{algpseudocode}
\usepackage[ngerman]{babel}

\usepackage{fancyvrb}
\usepackage[utf8]{inputenc}
\usepackage{amsmath}
\usepackage{amsthm}
\usepackage{amsfonts}
\usepackage{amssymb}
\usepackage{graphicx}

\usepackage{amssymb, amsmath}
\usepackage[utf8]{inputenc}
\usepackage{ngerman}
\usepackage{fancyhdr}
\usepackage{fullpage}
\usepackage{alltt}
\usepackage{soul}
\pagestyle{fancy}
\setlength{\headheight}{12.4pt}
\setlength{\headsep}{1.5\headheight}

% Ãœbungsblatt-Nummer eintragen
\newcommand{\AssignmentNumber}{2}

% 1. Person eintragen
\newcommand{\FirstAuthor}{LORENZ}
\newcommand{\FirstAuthorFirstName}{Peter}
\newcommand{\FirstAuthorMatnum}{}

% 2. Person eintragen
\newcommand{\SecAuthor}{ZIKO}
\newcommand{\SecAuthorFirstName}{Haris}
\newcommand{\SecAuthorMatnum}{}

%\newcommand{\ul}{\underline}
%\newcommand{\omega}{$\omega$}
%\newcommand{\omega'}{$\omega'$}
%\newcommand{\vert}{$\vert$}
%\newcommand{\delta}{$\delta$}
\newcommand{\AuthorFront}{{\normalsize
\begin{tabular}{|c|c|c|} \hline
    \textbf{Nachname} & \textbf{Vorname}       & \textbf{Matrikelnummer} \\ \hline \hline
    \FirstAuthor      & \FirstAuthorFirstName  & \FirstAuthorMatnum      \\ \hline
    \SecAuthor        & \SecAuthorFirstName    & \SecAuthorMatnum        \\ \hline
\end{tabular}}}

\author{\AuthorFront}
\newcommand{\Author}{\FirstAuthorMatnum, \SecAuthorMatnum}

\date{} % Kein Datum angegeben
\fancyfoot{} % Seitenzahl unten nicht anzeigen

\lhead{Excercise \AssignmentNumber}
\chead{\Author}
\rhead{Page \thepage}

\title{Software Paradigms SS 2015 \AssignmentNumber}

\newcommand{\letpi}{$\pi$}
\newcommand{\letomega}{$\omega$}

\newcommand{\InterA}{$I_{\mathcal{A}}$}
\newcommand{\InterP}{$I_{\mathcal{P}}$}
\newcommand{\InterT}{$I_{\mathcal{T}}$}

\newcommand{\TildeVar}{$\mathcal{~}$}

\newcommand{\unl}[1]{\underline{#1}}
%\newcommand{\math}{\mathcal{A}}

\begin{document}

\newcommand{\seccounter}{\addtocounter{section}{1} \thesection}
\newcommand{\beispiel}[2]{\subsection*{Exercise #1 \qquad}}
\maketitle
\thispagestyle{fancy}

%%%%%%%%%%%%%%%%%%%%%%%%%%%%%%%%%%%%%%%%%%%%%%%%%%%%%%%%%%%%%%%%%%%%%%%%%%%%%%%%%%%%%%%%%%%%%%%%%%%%%%%%

% Bsp 1 ------------------------------------------------------
\beispiel{\seccounter}{1}

\begin{verbatim}
     1 begin
     2 z := 1
     3 foreach i in [1,2,3] do 
     4   z := mul(z, i)
     5 end
    \end{verbatim} 
    
    Def.: formale Semantik für das Statement ($v \in IVS, t_1 \in T_L, a \in A$)
    \begin{enumerate}
     \item Assignements: $I_A(\omega,$ v \underline{:=} t) = w' $~_v$ $\omega$ und $w'(v)$=$I_T(\omega,t)$.
     \item Schleifen: $I_A(\omega,$ \underline{foreach} $t_i$ \underline{do} a) 
     = $I_A(I_A(\omega, a),$ \underline{foreach v do a}) wenn $I_T(\omega, t)$ mit t $\neq \varepsilon$.
     \item mul: definiert im Skriptum v. D. Gruss (S.29)
    \end{enumerate}

    $I_A (\omega$, \underline{begin z := 1; foreach v in [1,2,3] do z := mul(z,v)} \\
    = $I_A (I_A(\omega$, \underline{z := 1}), \underline{foreach v in [1,2,3] do} z:= mul(z,v)) \\

    NE: $\omega$\textsuperscript{1} $\sim_{\unl{x}}$ $\omega$ und berechnen $\omega^1$ (\underline{z}) = $I_T$=($\omega, $\underline{1}) = 1 \\
    
      = $I_A$ ($\omega$\textsuperscript{1}, \underline{foreach v do mul(z,v))} \\
      = mul($\omega$ \textsuperscript{1}(underline{z}), $\omega$ \textsuperscript{1}(underline{v})) = mul(1,1) = 1
    
    NR: $I_T $ ($\omega$\textsuperscript{1}, \underline{1}) und \underline{1} $\neq \varepsilon$ 
    
    
    = $I_A(I_A(\omega$\textsuperscript{1}, \underline{z := mul(z,v)}), foreach v do mul(z,v)) \\
    
    
    NE: $\omega$\textsuperscript{2} $\sim_{\unl{z}} \omega$ und berechnen $\omega^1$ (\underline{z}) = $I_T$=($\omega, $\underline{2}) = 2 \\
    = mul($\omega$ \textsuperscript{2}(underline{z}), $\omega$ \textsuperscript{2}(underline{v})) = mul(1,2) = 2
    
    = $I_A(I_A(\omega$\textsuperscript{2}, \underline{z := mul(z,v)}), \underline{foreach v do mul(z,v))} \\
    
     NE: $\omega$\textsuperscript{3} $\sim_{\unl{z}} \omega$ und berechnen $\omega^3$ (\underline{z}) = $I_T$=($\omega, $\underline{3}) = 3 \\
    = mul($\omega$ \textsuperscript{3}(underline{z}), $\omega$ \textsuperscript{3}(underline{v})) = mul(2,3) = 6
    
    = $I_A(I_A(\omega$\textsuperscript{3}, \underline{z := mul(z,v)}), \underline{foreach v do mul(z,v))} \\
    
    NE: $\omega$\textsuperscript{4} $\sim_{\unl{z}} \omega$ und berechnen $\omega^3$ (\underline{z}) = $I_T$=($\omega, \underline{\varepsilon}$) = $\varepsilon$ \\
    
Finished!
    


%\InterA

%\InterP

%\InterT

% Bsp 2 ------------------------------------------------------
\beispiel{\seccounter}{2}

Encoding for elements in A: \\
prove, that this is true: \\
\letpi((n,d)) = $build(n,build(d,[]))$ = $[n,d]$


% Bsp 3 ------------------------------------------------------
\beispiel{\seccounter}{2}

Encode for $f_{1}$:\vspace{0pt}\\
\letpi[insert](s, i) = if eq?(s, []) then build(i, []) else\vspace{2pt}\\
if eq?(first(s), i) then s else build(first(s), insert(rest(s), i))\vspace{2pt}\\
Dies ist eine rekursive Lösung, die als erstes Abfragt ob eine leere Menge gegeben ist, andererseits wird rekursiv überprüft, ob die Zahl i in der Menge mitenthalten ist. Wenn die Zahl nicht in der Liste ist, dann wir sie am Ende der List dazugehängt. Ansonsten wird die normale Liste zurückretuniert. \\

\letpi[insert](s, i) = if eq?(s, []) then [] else\vspace{2pt}\\
if eq?(first(s), i) then rest(s) else build(first(s), insert(rest(s), i))\vspace{2pt}\\
Dies ist eine rekursive Lösung, die als erstes Abfragt ob eine leere Menge gegeben ist, andererseits wird rekursiv überprüft, ob die Zahl i in der Menge mitenthalten ist. Wenn die Zahl in der Liste enthalten ist, dann wird sie von der Liste entfernt. Ansonsten wird die normale Liste zurückretuniert.\\

\letpi[isEmpty?](s) = if eq?(s, []) then T else F

Hier wird überprüft, ob die Liste leer ist oder nicht. Wenn sie leer ist, dann wird True zurückgegeben, ansonsten False.

\letpi[isElement?](s, i) = if eq?(s, []) then F else if eq?(first(s), i) then T else isElement?(rest(s), i)

\letpi((emptyS)) = if eq?(emptyS, []) then [] else emptyS

% Bsp 4 ------------------------------------------------------
\beispiel{\seccounter}{2}

1st: \unl{$(\forall x)(\exists y)$ eq?(mult(x,y),x)} 
\vspace{12pt} \\
\InterP(\letomega, \unl{$(\forall x)(\exists y)$ eq?(mult(x,y),x)}) \vspace{2pt} \\
$\iff$ $\forall \omega', \omega \sim_{\unl{x}} \omega' : $\InterP($\omega'$, \unl{$(\exists y)$ eq?(mult(x,y),x)}) \vspace{2pt} \\
$\iff$ $\forall \omega', \omega \sim_{\unl{x}} \omega' : \exists \omega'', \omega' \sim_{\unl{y}} \omega'' : $\InterP($\omega''$, \unl{eq?(mult(x,y),x)}) \vspace{2pt} \\
$\iff$ $\forall \omega', \omega \sim_{\unl{x}} \omega' : \exists \omega'', \omega' \sim_{\unl{y}} \omega'' : $eq?(\InterP($\omega''$, \unl{(mult(x,y),x)})) \vspace{2pt} \\
$\iff$ $\forall \omega', \omega \sim_{\unl{x}} \omega' : \exists \omega'', \omega' \sim_{\unl{y}} \omega'' :$ eq?(\InterP($\omega''$, \unl{(mult(x,y))}), \InterP($\omega''$, \unl{x})) \vspace{2pt} \\
$\iff$ $\forall \omega', \omega \sim_{\unl{x}} \omega' : \exists \omega'', \omega' \sim_{\unl{y}} \omega'' :$ eq?(mult(\InterP($\omega''$,\unl{x}),\InterP($\omega''$,\unl{y})), \InterP($\omega''$, \unl{x})) \vspace{2pt} \\
$\iff$ $\forall \omega', \omega \sim_{\unl{x}} \omega' : \exists \omega'', \omega' \sim_{\unl{y}} \omega'' :$ eq?(mult($\omega''$(\unl{x}),$\omega''$(\unl{y})),$\omega''$(\unl{x})) \vspace{12pt}

When we say, that $\omega''$(\unl{x}) = x and $\omega''$(\unl{y}) = y, then we can say that $x\cdot y=x$ only when y is 1.

We can say for $(\forall x)(\exists y) x \cdot y = x$ only true, if and only if $y = 1$.

Therefore we can say that: \\
$\iff T$ \\
is valid \\

2nd: \unl{$(\forall x)$eq?(build(x,nil),nil)$\lor$eq?(1,x)} \vspace{12pt} \\
$\iff$ \InterP($\omega$, \unl{$(\forall x)$eq?(build(x,nil),nil)$\lor$eq?(1,x)}) \vspace{2pt} \\
$\iff$ $\forall \omega',\omega \sim_{\unl{x}} \omega' :$ \InterP($\omega'$, \unl{eq?(build(x,nil),nil)$\lor$eq?(1,x)}) \vspace{2pt} \\
$\iff$ $\forall \omega',\omega \sim_{\unl{x}} \omega' :$ \InterP($\omega'$,\unl{eq?(build(x,nil),nil)}))$\lor$ \InterP($\omega'$,\unl{eq?(1,x)}) \vspace{2pt} \\
$\iff$ $\forall \omega',\omega \sim_{\unl{x}} \omega' :$ \InterP($\omega'$,\unl{eq?(build(x,nil),nil)}))$\lor$ eq?(\InterP($\omega'$,\unl{1}),\InterP($\omega'$,\unl{x})) \vspace{2pt} \\
$\iff$ $\forall \omega',\omega \sim_{\unl{x}} \omega' :$ eq?(\InterP($\omega'$,\unl{build(x,nil)}),\InterP($\omega'$,\unl{nil}))$\lor$ eq?(\InterP($\omega'$,\unl{1}),$\omega'$(\unl{x})) \vspace{2pt} \\
$\iff$ $\forall \omega',\omega \sim_{\unl{x}} \omega' :$ eq?(build(\InterP($\omega'$,\unl{x}),\InterP($\omega'$,\unl{nil})),\InterP($\omega'$,\unl{nil}))$\lor$ eq?($\omega'$(\unl{1}),$\omega'$(\unl{x})) \vspace{2pt}  \\
$\iff$ $\forall \omega',\omega \sim_{\unl{x}} \omega' :$ eq?(build(\InterP($\omega'$,\unl{x}),\InterP($\omega'$,\unl{nil})),$\omega'$(\unl{nil}))$\lor$ eq?($\omega'$(\unl{1}),$\omega'$(\unl{x})) \vspace{2pt} \\
$\iff$ $\forall \omega',\omega \sim_{\unl{x}} \omega' :$ eq?(build(\InterP($\omega'$,\unl{x}),$\omega'$(\unl{nil})),$\omega'$(\unl{nil}))$\lor$ eq?($\omega'$(\unl{1}),$\omega'$(\unl{x})) \vspace{2pt} \\
$\iff$ $\forall \omega',\omega \sim_{\unl{x}} \omega' :$ eq?(build($\omega'$(\unl{x}),$\omega'$(\unl{nil})),$\omega'$(\unl{nil}))$\lor$ eq?($\omega'$(\unl{1}),$\omega'$(\unl{x})) \vspace{6pt} \\
We can now say that $\omega'(\unl{x}) = x$ and $\omega'(nil) = []$, then we get \vspace{6pt} \\
$\iff$ $\forall \omega',\omega \sim_{\unl{x}} \omega' :$ eq?(build(x,[]),[])$\lor$eq?(1,x) \vspace{6pt} \\
When we write it more mathematically, it will look like \vspace{6pt} \\
$\forall x : ([x] = []) \lor 1 = x$ \vspace{6pt} \\
For $\forall$ we only need one contra example, e.g. $x = 2$ we can see that this make the whole statement False therefore we can say: \vspace{6pt} \\
$\iff F$ \\


3rd:($\forall x$)($\neg$ eq?(x,y) $\rightarrow$ gt?(x,y)) \dots Datentyp $N$

\InterP(\letomega, \unl{$(\forall x$)($\neg$ eq?(x,y) $\rightarrow$ gt?(x,y))}) \vspace{2pt} \\

$\iff$ $\forall \omega',\omega \sim_{\unl{x}} \omega'$ : \InterP($\omega'$,$\neg$ eq?(x,y) $\rightarrow$ gt?(x,y)))

$\iff$ $\forall \omega',\omega \sim_{\unl{x}} \omega'$ : \InterP($\omega'$, $\neg$ eq?(x,y)) $\rightarrow$ \InterP gt?(x,y))

$\iff$ $\forall \omega',\omega \sim_{\unl{x}} \omega'$ : $\neg$\InterP($\omega'$,  eq?(x,y)) $\rightarrow$ \InterP($\omega'$, gt?(x,y))

$\iff$ $\forall \omega$',$\omega \sim_{\unl{x}} \omega'$ : $\neg$ eq?(\InterP(x,y)) $\rightarrow$  gt?(\InterP(x,y))

$\iff$ $\forall \omega',\omega \sim_{\unl{x}} \omega'$ : $\neg$ eq?(x,y) $\rightarrow$ gt?(x,y)

\end{document}
