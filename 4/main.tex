\documentclass[10pt, a4paper]{article}

\usepackage{fancyhdr}
\usepackage{extramarks}
\usepackage{amsmath}
\usepackage{amsthm}
\usepackage{amsfonts}
\usepackage{tikz}
\usepackage[plain]{algorithm}
%\usepackage[euler]{textgreek}
\usepackage{algpseudocode}
\usepackage[ngerman]{babel}

\usepackage{fancyvrb}
\usepackage[utf8]{inputenc}
\usepackage{amsmath}
\usepackage{amsthm}
\usepackage{amsfonts}
\usepackage{amssymb}
\usepackage{graphicx}

\usepackage{amssymb, amsmath}
\usepackage[utf8]{inputenc}
\usepackage{ngerman}
\usepackage{fancyhdr}
\usepackage{fullpage}
\usepackage{alltt}
\usepackage{soul}
\usepackage{tabularx}
\pagestyle{fancy}
\setlength{\headheight}{12.4pt}
\setlength{\headsep}{1.5\headheight}

% Ãœbungsblatt-Nummer eintragen
\newcommand{\AssignmentNumber}{4}

% 1. Person eintragen
\newcommand{\FirstAuthor}{LORENZ}
\newcommand{\FirstAuthorFirstName}{Peter}
\newcommand{\FirstAuthorMatnum}{}

% 2. Person eintragen
\newcommand{\SecAuthor}{ZIKO}
\newcommand{\SecAuthorFirstName}{Haris}
\newcommand{\SecAuthorMatnum}{}

%\newcommand{\ul}{\underline}
%\newcommand{\omega}{$\omega$}
%\newcommand{\omega'}{$\omega'$}
%\newcommand{\vert}{$\vert$}
%\newcommand{\delta}{$\delta$}
\newcommand{\AuthorFront}{{\normalsize
\begin{tabular}{|c|c|c|} \hline

    \textbf{Nachname} & \textbf{Vorname}       & \textbf{Matrikelnummer} \\ \hline \hline
    \FirstAuthor      & \FirstAuthorFirstName  & \FirstAuthorMatnum      \\ \hline
    \SecAuthor        & \SecAuthorFirstName    & \SecAuthorMatnum        \\ \hline
\end{tabular}}}

\author{\AuthorFront}
\newcommand{\Author}{\FirstAuthorMatnum, \SecAuthorMatnum}

\date{} % Kein Datum angegeben
\fancyfoot{} % Seitenzahl unten nicht anzeigen

\lhead{Excercise \AssignmentNumber}
\chead{\Author}
\rhead{Page \thepage}

\title{Software Paradigms SS 2015 \AssignmentNumber}

\newcommand{\letpi}{$\pi$}
\newcommand{\letomega}{$\omega$}
\newcommand{\letTheta}{$\Theta$}

\newcommand{\InterA}{$I_{\mathcal{A}}$}
\newcommand{\InterP}{$I_{\mathcal{P}}$}
\newcommand{\InterT}{$I_{\mathcal{T}}$}

\newcommand{\pipe}{$\vert$\hphantom{i}}
\newcommand{\TildeVar}{$\mathcal{~}$}

\newcommand{\unl}[1]{\underline{#1}}
%\newcommand{\math}{\mathcal{A}}

% column with
\usepackage{array}
\newcolumntype{L}[1]{>{\raggedright\let\newline\\\arraybackslash\hspace{0pt}}m{#1}}
\newcolumntype{C}[1]{>{\centering\let\newline\\\arraybackslash\hspace{0pt}}m{#1}}
\newcolumntype{R}[1]{>{\raggedleft\let\newline\\\arraybackslash\hspace{0pt}}m{#1}}

\begin{document}

\newcommand{\seccounter}{\addtocounter{section}{1} \thesection}
\newcommand{\beispiel}[2]{\subsection*{Exercise #1 \qquad}}
\maketitle
\thispagestyle{fancy}

%%%%%%%%%%%%%%%%%%%%%%%%%%%%%%%%%%%%%%%%%%%%%%%%%%%%%%%%%%%%%%%%%%%%%%%%%%%%%%%%%%%%%%%%%%%%%%%%%%%%%%%%

% Bsp 1 ------------------------------------------------------
\beispiel{\seccounter}{1}

% 1.
\setcounter{secnumdepth}{1}
\subsubsection{1.}
t = p(Y,f(X),a), t' = p(X,f(b), X)\\

$\theta$ = \{X/Y\} \\
t = p(Y,f(Y), a); t' = p(Y,f(b),Y)\\

$\theta$ = $\theta$ $\cup$ \{Y/b\}\\
t = p(b,f(b),a); t' = p(b,f(b), b)\\

%$\theta$ = $\theta$ $\cup$ \{Y/a\} \\
%t = p(a,f(b),a); t' = p'(a,f(b),a) \\
%$\square$

%MGU = \{X/Y, X/b, Y/a\}

Not unifiable, see rule 2a from Slides 7 - S17. 
There are different values, and no Variables.


% 2.
\subsubsection{2.}

t = q(f(a),g(f(a)),Y), t' = q(X,g(X),f(g(a))) \\

\letTheta = \{X/f(a)\} \\
t = q(f(a),g(f(a)),Y), t' = q(f(a),g(f(a)),f(g(a))) \\

\letTheta = \letTheta $\cup$ \{Y/f(g(a))\} \\
t = q(f(a),g(f(a)),f(g(a))), t' = q(f(a),g(f(a)),f(g(a)))  \\
$\square$


MGU = \{X/f(a),Y/f(g(a))\}




% 3.
\subsubsection{3.}
t = r(b,f(g(X)), Z) und t' = r(X,Z,f(g(a))) \\

\letTheta = \{X/b\} \\
r(b,f(g(b)), Z); r(b,Z,f(g(a))) \\

\letTheta = \{Z/f(g(b))\} \\
r(b,f(g(b)), f(g(b))); r(b,f(g(b)),f(g(a))) \\

MGU = \{X/b, Z/f(g(b))\} \\


Not unifiable. Slides 7 - S17.
It exists a difference correspondig to 2.a, but there are not any variables. 
The algorithm stops, because there is a difference between t and t'.


% 4.
\subsubsection{4.}

t = s(X,c,Y), t' = s(g(a,b),Z,f(Y))\\

\letTheta = \{X/g(a,b)\} \\
t = s(g(a,b),c,Y), t' = s(g(a,b),Z,f(Y)) \\


Not unifiable, see rule 2b from Slides 7 - S17.
It is not possible to unify Y with f(Y).


% 5.
\subsubsection{5.}

t = u(f(Z,g(X)),Z), t' = u(f(g(a),g(Z)), X) \\

\letTheta = \{Z/g(a)\} \\
t = u(f(g(a),g(X)),g(a)), t' = u(f(g(a),g(g(a))), X)\\

\letTheta = \letTheta $\cup$ \{X/g(a)\} \\
t = u(f(g(a),g(g(a))),g(a)), t' = u(f(g(a),g(g(a))), g(a)) \\
$\square$ \\


MGU = \{Z/g(a),X/g(a)\}


% Bsp 2 ------------------------------------------------------
\beispiel{\seccounter}{2}

Predicate for greater(X,Y)

greater(s(\_), 0) :=.\\
greater(s(X),s(Y)) := greater(X,Y)

show, if the next statement is true or false:\\
:=greater(s(s(s(0))), s(s(s(0))))

$\neg$ greater(s(s(s(0))), s(s(s(0)))) \hphantom{00000} $\neg$ greater(X,Y) $\vee$ greater(s(X),s(Y))

\{X\pipe s(s(0)),Y\pipe s(s(0))\}\\

$\neg$ greater(s(s(0)), s(s(0))) \hphantom{00000} $\neg$ greater(X,Y) $\vee$ greater(s(X),s(Y))

\{X\pipe s(0),Y\pipe s(0)\}\\

$\neg$ greater(s(0), s(0)) \hphantom{00000} $\neg$ greater(X,Y) $\vee$ greater(s(X),s(Y))

\{X\pipe 0,Y\pipe 0\}\\

$\neg$ greater(0, 0) \hphantom{00000} $\neg$ greater(X,Y) $\vee$ greater(s(X),s(Y))

Not possible, because no MGU exists.\\

$\neg$ greater(0, 0) \hphantom{00000} $\neg$ greater(s(\_),0)

Not possible, because left side is not equal to right side.

Reject input => False

% Bsp 3 ------------------------------------------------------
\beispiel{\seccounter}{2}

% 1.
\setcounter{secnumdepth}{1}
\subsubsection{1.}
\begin{verbatim}
male(mannister).

male(lombard).

male(tytos).
female(genna).

male(jamballalia).

male(kevan).
male(lancel).
male(willem).
female(janei).

male(tywin).
male(jaime).
female(cersei).
male(tyrion).

parent(mannister, lombard).
parent(mannister, tytos).
parent(mannister, jamballalia).

parent(tytos, genna).
parent(tytos, tywin).
parent(tytos, kevan).

parent(kevan, lancel).
parent(kevan, willem).
parent(kevan, janei).

parent(tywin, jaime).
parent(tywin, cersei).
parent(tywin, tyrion).
\end{verbatim}



% 2.
\subsubsection{2.}
\begin{verbatim}
grandp(X,Y)  :- parent(X, P), parent(P, Y).
sibling(X,Y) :- parent(P,X), parent(P,Y), X\=Y.
cousin(X,Y)  :- parent(A,X), parent(B,Y), sibling(A,B), male(X).

\end{verbatim}
c1 = grandp(X,Y)  $\vee$ $\neg$parent(X, P) $\vee$ $\neg$parent(P, Y).\\
c2 = sibling(X,Y) $\vee$ $\neg$parent(P,X) $\vee$ $\neg$parent(P,Y) $\vee$ $\neg$X\=Y.\\
c3 = cousin(X,Y)  $\vee$ $\neg$ male(X) $\vee$ $\neg$parent(A,X) $\vee$ $\neg$parent(B,Y) $\vee$ $\neg$ sibling(A,B). \\

:= cousin(willem, jaime).\\

\begin{tabular}{L{4cm} |L{3cm} |L{5cm}}
Program & MGU & Rule  \\
\hline \hline
%----------------------------------------------------------------------------

$\neg$cousin(willem, jaime) &                                  & cousin(X,Y)  $\vee$ $\neg$ male(X) $\vee$ $\neg$parent(A,X) $\vee$ $\neg$parent(B,Y) $\vee$ $\neg$ sibling(A,B) \\

                            & $\theta$ = \{X/willem, Y/jaime\} & \\
\hline
%----------------------------------------------------------------------------
$\neg$ male(willem) $\vee$ $\neg$parent(A,willem) $\vee$ $\neg$parent(B,jaime) $\vee$ \newline $\neg$ sibling(A,B) 
                            &                                  & male(willem) \\
\hline
$\neg$parent(A,willem) $\vee$ $\neg$parent(B,jaime) $\vee$ \newline $\neg$ sibling(A,B) 
                            &                                  & parent(kevan, willem) and parent(tywin, jaime) \\
                            & $\theta$ = \{A/kevan, B/tywin\}  & \\
\hline
%----------------------------------------------------------------------------
sibling(kevan,tywing)         &                                & sibling(X,Y) $\vee$ $\neg$parent(P,X) $\vee$ $\neg$parent(P,Y) $\vee$ $\neg$X\=Y. \\
                             & $\theta$ = \{X/kevan, B/tywin\} & \\
\hline
%----------------------------------------------------------------------------
$\neg$parent(P,X) $\vee$ $\neg$parent(P,Y) $\vee$ $\neg$X\=Y & & parent(tytos, kevan) and parent(tytos, tywan) \\
                             & $\theta$ = \{P/tytos\} & \\

\hline
%----------------------------------------------------------------------------
$\square$ & & \\
\end{tabular}



% 3.
\subsubsection{3.}
granddaughter(X,Y) :- grandp(Y,X), female(X).

\begin{tabular}{L{4cm} |L{3cm} |L{5cm}}
Program & MGU & Rule  \\
\hline \hline
%----------------------------------------------------------------------------
:= granddaughter(X,tytos) & & granddaughter(X,Y) $\vee$ $\neg$grandp(Y,X) \newline $\vee$ female(X) \\
                          &  $\theta$ = \{Y/tytos\}               &   \\         \hline
$\neg$grandp(tytos,X) $\vee$ $\neg$female(X) &  & grandp(X,Y) $\vee$ $\neg$parent(X,P) $\vee$ $\neg$parent(P,Y) \\

                          & $\theta$ = \{X/tytos\}              & \\
\hline
$\neg$parent(tytos, P) $\vee$ $\neg$p(P,Y)&                   &  parents(tytos, kevan). \\
                          & $\theta$ = \{P/kevan\}            & \\
\hline
$\neg$parents(tytos, kevan) $\vee$ $\neg$parents(kevan,Y) &      & p(kevan, janei) \\
                          & $\theta$ = \{Y/janei\}            & \\
\hline
parents(kevan, janei)       &             & \\
$\square$  & &



\end{tabular}

% Bsp 4 ------------------------------------------------------
\beispiel{\seccounter}{2}

% 1.
\setcounter{secnumdepth}{1}
\subsubsection{1.}

edge(a, b, s(s(s(0)))).\\
edge(a, c, s(0)).\\
edge(c, d, s(s(0))).\\
edge(b, d, s(s(0))).\\\\
add(0,Y,Y).
add(s(X),Y,s(Z)) := add(X,Y,Z).
greaterequ(X,0).\\
greaterequ(s(X), s(Y)) := greaterequ(X,Y).\\\\
path(S,E,M) := find(S,E,N),greaterequ(M,N).\\\\
find(S,E,L) := edge(S,E,L).\\
find(S,E,N) := edge(S,X,L),find(X,E,N1),add(N1,L,N).

% 2.
\subsubsection{2.}

c1: edge(a, b, s(s(s(0)))).\\
c2: edge(a, c, s(0)).\\
c3: edge(c, d, s(s(0))).\\
c4: edge(b, d, s(s(0))).\\
c5: add(0,Y,Y).
c6: add(s(X),Y,s(Z)) $\vee$ $\neg$ add(X,Y,Z).
c7: greaterequ(X,0).\\
c8: greaterequ(s(X), s(Y)) $\vee$ $\neg$ greaterequ(X,Y).\\
c9: path(S,E,M) $\vee$ $\neg$ find(S,E,N) $\vee$ $\neg$ greaterequ(M,N).\\
c10: find(S,E,L) $\vee$ $\neg$ edge(S,E,L).\\
c11: find(S,E,N) $\vee$ $\neg$ edge(S,X,L) $\vee$ $\neg$ find(X,E,N1) $\vee$ $\neg$ add(N1,L,N)\\


\begin{tabular}{l|l|l}
Program & MGU & Rule  \\
\hline \hline

$\neg$ path(a,d,s(s(s(s(0))))) & \{S/a, E/d, M/s(s(s(s(0))))\} & c9: path(S,E,M) \\
&&$\vee$ $\neg$ find(S,E,N) \\
&&$\vee$ $\neg$ greaterequ(M,N) \\
\hline

$\neg$ find(a,d,N) $\vee$ & \{S/a,E/d\}& c11: find(S,E,N)\\
$\neg$ greaterequ(s(s(s(s(0)))),N) && $\vee$ $\neg$ edge(S,X,L)\\
&& $\vee$ $\neg$ find(X,E,N1) \\
&& $\vee$ $\neg$ add(N1,L,N)\\
\hline

$\neg$ edge(a,X,L) $\vee$ & \{X/c,L/s(0)\}& c2: edge(a, c, s(0))\\
$\neg$ find(X,d,N1) $\vee$ && \\
$\neg$ add(N1,L,N) && \\
$\neg$ greaterequ(s(s(s(s(0)))),N) && \\
\hline

$\neg$ find(c,d,N1) $\vee$ & \{S/c,E/d/L/N1\} & c10: find(S,E,L) \\
$\neg$ add(N1,s(0),N) && $\vee$ $\neg$ edge(S,E,L) \\
$\neg$ greaterequ(s(s(s(s(0)))),N) && \\
\hline

$\neg$ edge(c,d,N1) $\vee$ & \{N1/s(s(0))\} & c3: edge(c, d, s(s(0))) \\
$\neg$ add(N1,s(0),N) &&\\
$\neg$ greaterequ(s(s(s(s(0)))),N) && \\
\hline

$\neg$ add(s(s(0)),s(0),N) $\vee$ & \{X/s(0),Y/s(0),N/s(Z)\} & c6: add(s(X),Y,s(Z)) \\
$\neg$ greaterequ(s(s(s(s(0)))),N) && $\vee$ $\neg$ add(X,Y,Z) \\
\hline

$\neg$ add(s(0),s(0),s(Z)) $\vee$ & \{X'/s(0),Y'/s(0),Z/s(Z')\} & c6: add(s(X'),Y',s(Z')) \\
$\neg$ greaterequ(s(s(s(s(0)))),s(Z)) && $\vee$ $\neg$ add(X',Y',Z') \\
\hline

$\neg$ add(0,s(0),s(s(Z'))) $\vee$ & \{Y''/s(0),Z'/Y''\} & c5: add(0,Y'',Y'') \\
$\neg$ greaterequ(s(s(s(s(0)))),s(s(Z'))) && \\
\hline

$\neg$ greaterequ(s(s(s(s(0)))),s(s(s(0)))) & \{X/s(s(s(0))),Y/s(s(s(0)))\} & c8: greaterequ(s(X), s(Y)) \\
&& $\vee$ $\neg$ greaterequ(X,Y) \\
\hline

$\neg$ greaterequ(s(s(s(0))),s(s(0))) & \{X'/s(s(0)),Y'/s(s(0))\} & c8: greaterequ(s(X'), s(Y')) \\
&& $\vee$ $\neg$ greaterequ(X',Y') \\
\hline

$\neg$ greaterequ(s(s(0)),s(0)) & \{X''/s(0),Y''/s(0)\} & c8: greaterequ(s(X''), s(Y'')) \\
&& $\vee$ $\neg$ greaterequ(X'',Y'') \\
\hline

$\neg$ greaterequ(s(0),0) & \{X'''/s(0)\} & c7: greaterequ(X''',0) \\
\hline

$\square$ & &  \\
\hline

\end{tabular}

We can say, that the input 

% Bsp 5 ------------------------------------------------------
\beispiel{\seccounter}{2}

% 1.
\setcounter{secnumdepth}{1}
\subsubsection{1.}

less(0, s(Y)).\\
less(s(X),s(Y)) := less(X,Y).\\\\
union(null,null,null).\\
union(null,Y,Y).\\
union(X,null,X).\\\\
union(build(X,Xs),null,build(X,Zs)) := union(Xs,null,Zs).\\
union(null,build(Y,Ys),build(Y,Zs)) := union(null,Ys,Zs).\\\\
union(build(Z,Xs),build(Z,Ys),build(Z,Zs)) := union(Xs,Ys,Zs).\\
union(build(X,Xs),build(Y,Ys),build(X,Zs)) := less(X,Y), union(Xs,build(Y,Ys),Zs).\\
union(build(X,Xs),build(Y,Ys),build(Y,Zs)) := less(Y,X), union(build(X,Xs),Ys,Zs).\\

% 2.
\subsubsection{2.}

\begin{tabular}{L{1.5cm} | L{13.5cm}}

& Resolution for Example 5 Nr. 2 \\
\hline \hline
Program &
:= union(build(0, build(s(s(0)), null)), build(s(0) , null), build(0, build(s(0), build(s(s(0)), null)))).\\
\hline
Rule &
union(build(X,XS),build(Y,Ys),build(X,Zs)) := less(X,Y), union(XS,build(Y,YS),Zs).\\
\hline
MGU &
\{X/0,Xs/build(s(s(0)), null),Y/s(0),Ys/null,Zs/build(s(0), build(s(s(0)), null))\}\\
\hline\hline

Program &
:= less(0,s(0)), union(build(s(s(0)), null),build(s(0),null),build(s(0), build(s(s(0)), null))).\\
\hline
Rule &
less(0, s(Y)).\\
\hline
MGU &
\{Y/0\}\\
\hline\hline

Program &
:= union(build(s(s(0)), null),build(s(0),null),build(s(0), build(s(s(0)), null))).\\
\hline
Rule &
union(build(X,Xs),build(Y,Ys),build(Y,Zs)) := less(Y,X), union(build(X,Xs),Ys,Zs).\\
\hline
MGU &
\{X/s(s(0)),Xs/null,Y/s(0),Ys/null,Zs/build(s(s(0)), null)\}\\
\hline\hline

Program &
:= less(s(0),s(s(0))), union(build(s(s(0)),null),null,build(s(s(0)), null)).\\
\hline
Rule &
less(s(X),s(Y)) := less(X,Y).\\
\hline
MGU &
\{X/0,Y/s(0)\}\\
\hline\hline

Program &
:= less(0,s(0)), union(build(s(s(0)),null),null,build(s(s(0)), null)).\\
\hline
Rule &
less(0, s(Y)).\\
\hline
MGU &
\{Y/0\}\\
\hline\hline

Program &
:= union(build(s(s(0)),null),null,build(s(s(0)), null)).\\
\hline
Rule &
union(build(X,Xs),null,build(X,Zs)) := union(Xs,null,Zs).\\
\hline
MGU &
\{X/s(s(0)),Xs/null,Zs/null\}\\
\hline\hline

Program &
:= union(null,null,null).\\
\hline
Rule &
union(null,null,null).\\
\hline
MGU &
\{\}\\
\hline\hline

Program &
$\square$\\
\hline

\end{tabular}

\end{document}
